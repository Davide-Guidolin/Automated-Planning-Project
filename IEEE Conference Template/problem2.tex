The second problem required to think about the \textbf{move} action in a different persective, 
ensuring that a robot can load the boxes on a carrier instead of itself, and that the carrier can be transported by a robot to the destinations.\\
The solution we developed are two, the first one is simpler since it only has an additional type \textbf{carrier} and its \textbf{capacity} is represented through a function.
The initial capacity of each carrier can be specified in the problem, and it is managed through the usage of fluents: \textbf{load} can be performed only if the capacity is greater than 0, and the effect is to decrement it, the opposite for \textbf{unload}.\\
Another solution has been implemented since the first one relies on the \textbf{fluents} requirement, which is not supported by many planner we used.
We took inspiration by an example provided during the lab session, in which a new type called \textbf{capacity-number} is used to described all the possible capacities.
Then for each \textbf{load} and \textbf{unload} operation done on the carrier is important not to forget to specify the precondition \textbf{capacity-predecessor} through which we ensure the numbers we use to upload the capacity are consequent.
The order of the capacity is specified in the problem as follows:
\begin{verbatim}
(capacity_predecessor capacity_0 capacity_1)
(capacity_predecessor capacity_1 capacity_2)
(capacity_predecessor capacity_2 capacity_3)
(capacity_predecessor capacity_3 capacity_4)
\end{verbatim}
Plus the initial capacity of each carrier (4 as requested):
\begin{verbatim}
    (capacity carrier1 capacity_4)
\end{verbatim}
Finally the last modification with respect to the first problem is the \textbf{move} action, which no longer allow a robot to carry a box, but a carrier instead.
Thanks to this any robot can carry any carrier, even if loaded by another robot.\\
To test it with a more complex environment we increased the number of people to 6, the boxes to 5, the locations to 8 and the supplies to 5, the robot and the carrier are just one each, as requested in the assignement.
Of course there are no constraints neither in the number of robots nor of carriers.
From this point on, all next problems are solved using solution 2 described above as base.

The problem definition without fluents is reported in \textit{Fig. \ref{problem2_problem}} while the obtained plan is reported in \textit{Fig. \ref{problem2_plan}}.
The problem definition using fluents is reported in \textit{Fig. \ref{problem2_problem_fluents}} while the obtained plan is reported in \textit{Fig. \ref{problem2_plan_fluents}}.