The following report aims to give a detailed explanation of the solutions of the 5 problems proposed in the assignment.
The problems considered a scenario where one or more robots have to deliver emergency supplies to injured people located in space.
These were the initial explicit assumptions:
\begin{itemize}
    \item Each injured person is at a specific location, and does not move.
    \item Each box is initially at a specific location and can be filled with a specific content if empty. Box contents shall be modeled in a generic way, so that new contents can easily be introduced in the problem instance.
    \item Each person either has or does not have a box with a specific content.
    \item There can be more than one person at any given location.
    \item The robotic agents can move directly between arbitrary locations.
    \item We want to be able to expand this domain for multiple robotic agents in the future.
\end{itemize}  
While the suggested actions were the following:
\begin{itemize}
    \item Fill a box with a content. Robot box and content must be in the same place, box must be empty.
    \item Empty a box, satisfying the person at that specific location.
    \item Pick up a box located in the same place as the robot.
    \item Move to another location.
    \item Deliver a box.
\end{itemize}

Based on these assumptions here is the formulated solution used in all problems.
We considered 5 main types:
\begin{itemize}
    \item Robot
    \item Person
    \item Location
    \item Box
    \item Content
\end{itemize}
and 5 main actions:
\begin{itemize}
    \item Fill 
    \item Empty
    \item Load
    \item Move
    \item Unload
\end{itemize}

The main workflow the robot follow to satify a person is to \textbf{fill} a box with a content,
\textbf{load} the box on itself and \textbf{move} carrying the box toward the destination.
Once it gets where the person is located it \textbf{unloads} the box and \textbf{empties} it giving the content to the person.
After that the robot needs to restart in order to satisfy another person, so it can re-use the same box as before and carry it to the place where the supplies are stored, or look for another box.
We assume the supplies are infinite where stored, but only one portion can fit in a box, so if 2 people need the same supply the robot must re-fill with it.
\\
For doing so the folllowing informations are needed:
\begin{itemize}
    \item Where a robot is (\textbf{robot-at}).
    \item If a robot is carrying a box(\textbf{robot-loaded}).
    \item Which box a robot is carrying (\textbf{robot-has}).
    \item If a box is full (\textbf{box-full}).
    \item Where a box is (\textbf{box-at}).
    \item What is the content of a box (\textbf{box-contains}).
    \item Where a supply is stored (\textbf{content-at}); specified in the initial assumptions of the problem and fixed.
    \item Where a person is (\textbf{person-at}); specified in the initial assumptions of the problem and fixed.
    \item What supply a person has (\textbf{person-has}); which represent the goal.
\end{itemize}
For all problems we used \textbf{negative preconditions} since the predicates make more sense this way and it is supported by almost every planner.
We also made a version without \textbf{negative precondition} that has very few differences, it is sufficient to use \textbf{box-empty} and \textbf{robot-unloaded} and remove the negated preconditions from the \textbf{move} action. 
Notice that this configuration ensure there are no constraints on the number of robots or boxes in order to satisfy any number of people.