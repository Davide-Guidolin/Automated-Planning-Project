The main workflow to solve all the problems was the following: 
\begin{itemize}
    \item Understand the assignment text, including environment constraints and goals.
    \item Formulate an idea of the environment as general as possible, in order to include also all the sub-cases.
    \item Write the PDDL domain and problem files with a goal that should be achieved by few actions.
    \item Enlarge the goal with a bottom-up approach to finally reach the final goal, solving the issues encountered at each step. 
\end{itemize}
Following this simple steps we managed to solve all the problems in little time.
A big issue that has been encountered and delayed the work was finding the correct planner and set everything correctly: 
we found that many planners don't support simple requirements such as "fluents" or "negative-precondition", so we had to try many and to develop different
solutions in order to finally choose not only the best one, but the one with less compatibility problems.\\
This is the case of problem 2 and problem 5, where two different solutions are proposed for each one.
Problem 5 was in particular the most problematic since the deafult planner do not support negative preconditions, and we took a while to find out this was the issue that made it not work,
since the informations given by the planner concerning the errors are never exhaustive.\\
An interesting observation is that if no optimization is requested, different planners may take very different path to reach the goal, sometimes the difference is of several actions.
This can be noticed especially when dealing with those problems implying the capacity of the carrier, and so the possibility to carry many boxes at the same time.
The heuristic of some planners, in fact, made them repeat the \textbf{load} action many times in a row exploiting the full capacity of the carrier and resulting in less trips of the robot,
whereas others prefered to move as soon as possible to the satisfaction of every single goal resulting in more trips and a less efficient plan.
We recognize this is not a real issue since no optimization was requested, but it's worth to notice that in a general search case heuristics can make a big difference.
Another big difference found among different planners that can be attributed to the heuristics is the search time needed to find a plan.\\
It could be interesting to evaluate the performances of different planners requesting an optimization in the plan.